% Options for packages loaded elsewhere
\PassOptionsToPackage{unicode}{hyperref}
\PassOptionsToPackage{hyphens}{url}
\PassOptionsToPackage{dvipsnames,svgnames,x11names}{xcolor}
%
\documentclass[
]{article}
\usepackage{amsmath,amssymb}
\usepackage{lmodern}
\usepackage{iftex}
\ifPDFTeX
  \usepackage[T1]{fontenc}
  \usepackage[utf8]{inputenc}
  \usepackage{textcomp} % provide euro and other symbols
\else % if luatex or xetex
  \usepackage{unicode-math}
  \defaultfontfeatures{Scale=MatchLowercase}
  \defaultfontfeatures[\rmfamily]{Ligatures=TeX,Scale=1}
\fi
% Use upquote if available, for straight quotes in verbatim environments
\IfFileExists{upquote.sty}{\usepackage{upquote}}{}
\IfFileExists{microtype.sty}{% use microtype if available
  \usepackage[]{microtype}
  \UseMicrotypeSet[protrusion]{basicmath} % disable protrusion for tt fonts
}{}
\makeatletter
\@ifundefined{KOMAClassName}{% if non-KOMA class
  \IfFileExists{parskip.sty}{%
    \usepackage{parskip}
  }{% else
    \setlength{\parindent}{0pt}
    \setlength{\parskip}{6pt plus 2pt minus 1pt}}
}{% if KOMA class
  \KOMAoptions{parskip=half}}
\makeatother
\usepackage{xcolor}
\usepackage[margin=1in]{geometry}
\usepackage{color}
\usepackage{fancyvrb}
\newcommand{\VerbBar}{|}
\newcommand{\VERB}{\Verb[commandchars=\\\{\}]}
\DefineVerbatimEnvironment{Highlighting}{Verbatim}{commandchars=\\\{\}}
% Add ',fontsize=\small' for more characters per line
\usepackage{framed}
\definecolor{shadecolor}{RGB}{248,248,248}
\newenvironment{Shaded}{\begin{snugshade}}{\end{snugshade}}
\newcommand{\AlertTok}[1]{\textcolor[rgb]{0.94,0.16,0.16}{#1}}
\newcommand{\AnnotationTok}[1]{\textcolor[rgb]{0.56,0.35,0.01}{\textbf{\textit{#1}}}}
\newcommand{\AttributeTok}[1]{\textcolor[rgb]{0.77,0.63,0.00}{#1}}
\newcommand{\BaseNTok}[1]{\textcolor[rgb]{0.00,0.00,0.81}{#1}}
\newcommand{\BuiltInTok}[1]{#1}
\newcommand{\CharTok}[1]{\textcolor[rgb]{0.31,0.60,0.02}{#1}}
\newcommand{\CommentTok}[1]{\textcolor[rgb]{0.56,0.35,0.01}{\textit{#1}}}
\newcommand{\CommentVarTok}[1]{\textcolor[rgb]{0.56,0.35,0.01}{\textbf{\textit{#1}}}}
\newcommand{\ConstantTok}[1]{\textcolor[rgb]{0.00,0.00,0.00}{#1}}
\newcommand{\ControlFlowTok}[1]{\textcolor[rgb]{0.13,0.29,0.53}{\textbf{#1}}}
\newcommand{\DataTypeTok}[1]{\textcolor[rgb]{0.13,0.29,0.53}{#1}}
\newcommand{\DecValTok}[1]{\textcolor[rgb]{0.00,0.00,0.81}{#1}}
\newcommand{\DocumentationTok}[1]{\textcolor[rgb]{0.56,0.35,0.01}{\textbf{\textit{#1}}}}
\newcommand{\ErrorTok}[1]{\textcolor[rgb]{0.64,0.00,0.00}{\textbf{#1}}}
\newcommand{\ExtensionTok}[1]{#1}
\newcommand{\FloatTok}[1]{\textcolor[rgb]{0.00,0.00,0.81}{#1}}
\newcommand{\FunctionTok}[1]{\textcolor[rgb]{0.00,0.00,0.00}{#1}}
\newcommand{\ImportTok}[1]{#1}
\newcommand{\InformationTok}[1]{\textcolor[rgb]{0.56,0.35,0.01}{\textbf{\textit{#1}}}}
\newcommand{\KeywordTok}[1]{\textcolor[rgb]{0.13,0.29,0.53}{\textbf{#1}}}
\newcommand{\NormalTok}[1]{#1}
\newcommand{\OperatorTok}[1]{\textcolor[rgb]{0.81,0.36,0.00}{\textbf{#1}}}
\newcommand{\OtherTok}[1]{\textcolor[rgb]{0.56,0.35,0.01}{#1}}
\newcommand{\PreprocessorTok}[1]{\textcolor[rgb]{0.56,0.35,0.01}{\textit{#1}}}
\newcommand{\RegionMarkerTok}[1]{#1}
\newcommand{\SpecialCharTok}[1]{\textcolor[rgb]{0.00,0.00,0.00}{#1}}
\newcommand{\SpecialStringTok}[1]{\textcolor[rgb]{0.31,0.60,0.02}{#1}}
\newcommand{\StringTok}[1]{\textcolor[rgb]{0.31,0.60,0.02}{#1}}
\newcommand{\VariableTok}[1]{\textcolor[rgb]{0.00,0.00,0.00}{#1}}
\newcommand{\VerbatimStringTok}[1]{\textcolor[rgb]{0.31,0.60,0.02}{#1}}
\newcommand{\WarningTok}[1]{\textcolor[rgb]{0.56,0.35,0.01}{\textbf{\textit{#1}}}}
\usepackage{graphicx}
\makeatletter
\def\maxwidth{\ifdim\Gin@nat@width>\linewidth\linewidth\else\Gin@nat@width\fi}
\def\maxheight{\ifdim\Gin@nat@height>\textheight\textheight\else\Gin@nat@height\fi}
\makeatother
% Scale images if necessary, so that they will not overflow the page
% margins by default, and it is still possible to overwrite the defaults
% using explicit options in \includegraphics[width, height, ...]{}
\setkeys{Gin}{width=\maxwidth,height=\maxheight,keepaspectratio}
% Set default figure placement to htbp
\makeatletter
\def\fps@figure{htbp}
\makeatother
\setlength{\emergencystretch}{3em} % prevent overfull lines
\providecommand{\tightlist}{%
  \setlength{\itemsep}{0pt}\setlength{\parskip}{0pt}}
\setcounter{secnumdepth}{-\maxdimen} % remove section numbering
\usepackage[french]{babel}
\ifLuaTeX
  \usepackage{selnolig}  % disable illegal ligatures
\fi
\IfFileExists{bookmark.sty}{\usepackage{bookmark}}{\usepackage{hyperref}}
\IfFileExists{xurl.sty}{\usepackage{xurl}}{} % add URL line breaks if available
\urlstyle{same} % disable monospaced font for URLs
\hypersetup{
  pdfauthor={Vincent Runge},
  colorlinks=true,
  linkcolor={Maroon},
  filecolor={Maroon},
  citecolor={Blue},
  urlcolor={blue},
  pdfcreator={LaTeX via pandoc}}

\title{Evaluation des algorithmes heuristiques pour le voyageur de
commerce avec les méthodes d'insertion\\
\includegraphics[width=1in,height=\textheight]{Images/logo_lamme.png}
\includegraphics[width=1.7in,height=\textheight]{Images/logo_UEVE.png}\\
\hspace*{0.333em}M2 Data Science Algorithmique\\
Exemple de projet}
\author{Vincent Runge}
\date{jeudi 27 octobre 2022}

\begin{document}
\maketitle

{
\hypersetup{linkcolor=}
\setcounter{tocdepth}{2}
\tableofcontents
}
\noindent\hrulefill

Librairies à installer:

\begin{Shaded}
\begin{Highlighting}[]
\FunctionTok{library}\NormalTok{(ggplot2) }\CommentTok{\#ggplot}
\FunctionTok{library}\NormalTok{(reshape2) }\CommentTok{\#melt}
\FunctionTok{library}\NormalTok{(parallel) }\CommentTok{\#mclapply}
\end{Highlighting}
\end{Shaded}

La librairie de ce travail est disponible sur github:

\begin{Shaded}
\begin{Highlighting}[]
\CommentTok{\#devtools::install\_github("vrunge/TSP")}
\FunctionTok{library}\NormalTok{(TSP)}
\end{Highlighting}
\end{Shaded}

\hypertarget{description-du-probluxe8me-et-objectif}{%
\section{Description du problème et
objectif}\label{description-du-probluxe8me-et-objectif}}

On tire de manière aléatoire dans le carré unité selon une loi uniforme
\(\mathcal{U}(0,1) \times \mathcal{U}(0,1)\) un nombre \texttt{n} de
villes. On donne ici un exemple avec 40 villes

\begin{Shaded}
\begin{Highlighting}[]
\NormalTok{n }\OtherTok{\textless{}{-}} \DecValTok{40}
\NormalTok{villes }\OtherTok{\textless{}{-}} \FunctionTok{matrix}\NormalTok{(}\FunctionTok{runif}\NormalTok{(}\DecValTok{2}\SpecialCharTok{*}\NormalTok{n), n, }\DecValTok{2}\NormalTok{)}
\end{Highlighting}
\end{Shaded}

\includegraphics{rapport_TSP_files/figure-latex/plot example-1.pdf}

Notre premier objectif est de construire un ``plus court chemin'' par un
\textbf{algorithme heuristique}. On comparera différentes
\textbf{méthodes d'insertion} et on analysera leur \textbf{temps de
calcul} numériquement.

\textbf{Nos objectifs : }

\begin{itemize}
\item
  comparer les performances en temps des différents algorithmes
\item
  évaluer la distance à la solution optimale
\item
  pour cela coder un algorithme exact (par exemple \emph{branch and
  bound}).
\end{itemize}

On note \(c(i,j)\) le coût pour passer de la ville \(i\) à la ville
\(j\). Notre objectif est de trouver la permutation des indices
\((1,...,n)\) notée \((v_1,...,v_n)\) qui minimisera la longueur du
tour:

\[\sum_{i=1}^{n}c(v_i,v_{i+1})\] avec \(v_{n+1} = v_1\) (pour revenir à
la ville de départ). Remarquez bien qu'une permutation contient une et
une seule fois chaque indice de sorte que le tour est complet et passe
bien par chaque ville une et une seule fois.

\hypertarget{lalgorithme-nauxeff-du-plus-proche-voisin}{%
\section{L'algorithme naïf du plus proche
voisin}\label{lalgorithme-nauxeff-du-plus-proche-voisin}}

C'est la méthode la plus simple. Elle consiste à partir d'une ville
\(i\) et de contruire le chemin de proche en proche en ajoutant en bout
de chemin la ville la plus proche parmi les villes non explorées. Quand
toutes les villes sont explorées on revient à la première ville pour
fermer le tour.

On peut répéter la procédure pour chaque ville de départ (on exécute
ainsi \(n\) fois cette méthode) pour choisir le meilleur chemin parmi
les \(n\) obtenus.

Exemple avec une seule ville de départ:

\begin{Shaded}
\begin{Highlighting}[]
\NormalTok{res1 }\OtherTok{\textless{}{-}} \FunctionTok{NN\_TSP}\NormalTok{(villes)}
\FunctionTok{plot}\NormalTok{(}\AttributeTok{x =}\NormalTok{ res1, }\AttributeTok{data =}\NormalTok{ villes)}
\end{Highlighting}
\end{Shaded}

\includegraphics{rapport_TSP_files/figure-latex/plus proche voisin-1.pdf}

\begin{Shaded}
\begin{Highlighting}[]
\NormalTok{(t1 }\OtherTok{\textless{}{-}} \FunctionTok{tour\_length}\NormalTok{(res1, villes))}
\end{Highlighting}
\end{Shaded}

\begin{verbatim}
## [1] 6.762322
\end{verbatim}

On remarque que la fermeture du tour est assez peu optimale\ldots{}

\textbf{EXERCICE :} pour le problème euclidien du voyageur de commerce
(inégalité triangulaire respectée), le tour optimal ne peut pas contenir
de croisement. \textbf{Le prouver!}

\hypertarget{les-algorithmes-dinsertion}{%
\section{Les algorithmes d'insertion}\label{les-algorithmes-dinsertion}}

les algorithmes d'insertion consistent à \textbf{insérer les villes
l'une après l'autre dans un tour partiel} (contenant qu'un sous-ensemble
des villes) partant d'une ou deux villes de départ.

\hypertarget{lalgorithme-dinsertion-cheapest-le-moins-cher}{%
\subsection{\texorpdfstring{L'algorithme d'insertion \emph{cheapest}
(``le moins
cher'')}{L'algorithme d'insertion cheapest (``le moins cher'')}}\label{lalgorithme-dinsertion-cheapest-le-moins-cher}}

Pour un tour partiel déjà constitué on cherche l'arrête (le couple de
villes) \((i,j)\) et la ville encore non incluse \(k\) qui minimise la
quantité

\[c(i,k) + c(k,j) - c(i,j)\] C'est ainsi l'insertion la moins coûteuse.
On pourra aussi répéter l'algorithme pour chacune des villes de départ.

Un exemple avec une seule ville de départ:

\begin{Shaded}
\begin{Highlighting}[]
\NormalTok{res2 }\OtherTok{\textless{}{-}} \FunctionTok{greedy\_TSP\_best}\NormalTok{(villes)}
\FunctionTok{plot}\NormalTok{(}\AttributeTok{x =}\NormalTok{ res2, }\AttributeTok{data =}\NormalTok{ villes)}
\end{Highlighting}
\end{Shaded}

\includegraphics{rapport_TSP_files/figure-latex/insert cheapest-1.pdf}

\begin{Shaded}
\begin{Highlighting}[]
\NormalTok{(t2 }\OtherTok{\textless{}{-}} \FunctionTok{tour\_length}\NormalTok{(res2, villes))}
\end{Highlighting}
\end{Shaded}

\begin{verbatim}
## [1] 5.625291
\end{verbatim}

\hypertarget{lalgorithme-dinsertion-nearest-le-plus-proche}{%
\subsection{\texorpdfstring{L'algorithme d'insertion \emph{nearest}
(``le plus
proche'')}{L'algorithme d'insertion nearest (``le plus proche'')}}\label{lalgorithme-dinsertion-nearest-le-plus-proche}}

Pour un tour partiel déjà constitué on cherche la ville \(i\) et la
ville encore non incluse \(k\) qui minimise la quantité \(c(i,k)\) :
c'est \textbf{la ville la plus proche du tour}. Une fois trouvée on
insert cette ville à sa position optimale en trouvant l'arrête \((i,j)\)
qui minimise \(c(i,k) + c(k,j) - c(i,j)\) C'est ainsi l'insertion la
plus proche. On pourra aussi répéter l'algorithme pour chacune des
villes de départ.

Un exemple avec une seule ville de départ:

\begin{Shaded}
\begin{Highlighting}[]
\NormalTok{res3 }\OtherTok{\textless{}{-}} \FunctionTok{greedy\_TSP\_min}\NormalTok{(villes)}
\FunctionTok{plot}\NormalTok{(}\AttributeTok{x =}\NormalTok{ res3, }\AttributeTok{data =}\NormalTok{ villes)}
\end{Highlighting}
\end{Shaded}

\includegraphics{rapport_TSP_files/figure-latex/insert nearest-1.pdf}

\begin{Shaded}
\begin{Highlighting}[]
\NormalTok{(t3 }\OtherTok{\textless{}{-}} \FunctionTok{tour\_length}\NormalTok{(res3, villes))}
\end{Highlighting}
\end{Shaded}

\begin{verbatim}
## [1] 6.249731
\end{verbatim}

\hypertarget{lalgorithme-dinsertion-farthest-le-plus-uxe9loignuxe9}{%
\subsection{\texorpdfstring{L'algorithme d'insertion \emph{farthest}
(``le plus
éloigné'')}{L'algorithme d'insertion farthest (``le plus éloigné'')}}\label{lalgorithme-dinsertion-farthest-le-plus-uxe9loignuxe9}}

Pour un tour partiel déjà constitué on cherche pour chaque ville non
encore incluse \(k\), la ville \(i\) du tour la plus proche. On obtient
des distances \(c(i,k)\) avec autant de couples \((i,k)\) qu'il y a de
villes non incluses. On sélectionne le plus grande de ces distances et
la ville \(k\) qui lui est associée. On insère cette ville \(k\) à sa
position optimale selon le critère habituel (min de
\(c(i,k) + c(k,j) - c(i,j)\)).

Un exemple avec une seule ville de départ:

\begin{Shaded}
\begin{Highlighting}[]
\NormalTok{res4 }\OtherTok{\textless{}{-}} \FunctionTok{greedy\_TSP\_max}\NormalTok{(villes)}
\FunctionTok{plot}\NormalTok{(}\AttributeTok{x =}\NormalTok{ res4, }\AttributeTok{data =}\NormalTok{ villes)}
\end{Highlighting}
\end{Shaded}

\includegraphics{rapport_TSP_files/figure-latex/insert farthest-1.pdf}

\begin{Shaded}
\begin{Highlighting}[]
\NormalTok{(t4 }\OtherTok{\textless{}{-}} \FunctionTok{tour\_length}\NormalTok{(res4, villes))}
\end{Highlighting}
\end{Shaded}

\begin{verbatim}
## [1] 5.27257
\end{verbatim}

Affichés tous ensemble :

\includegraphics{rapport_TSP_files/figure-latex/plot all-1.pdf}

Il est possible dans ce cadre eucliden d'obtenir une
\href{http://www.cs.albany.edu/~res/tsp_sicomp_1977.pdf}{bornes sur la
longueur du tour}. \textbf{Ces algorithmes heuristiques sont donc des
algorithmes d'approximation} (sauf peut-être pour \emph{farthest})!

\[algo(cheapest) \le 2 \,algo(opt)\]

\[algo(closest) \le 2 \,algo(opt)\]

\[algo(farthest) \le (\lceil log_2(n) \rceil + 1) \,algo(opt)\]

\hypertarget{comparaison-des-performances}{%
\section{Comparaison des
performances}\label{comparaison-des-performances}}

\hypertarget{pour-les-diffuxe9rents-algorithmes-heuristiques}{%
\subsection{Pour les différents algorithmes
heuristiques}\label{pour-les-diffuxe9rents-algorithmes-heuristiques}}

On répète 100 fois les 4 algorithmes sur des données générées par
\(\mathcal{U}[0,1] \times \mathcal{U}[0,1]\)

\begin{verbatim}
## No id variables; using all as measure variables
\end{verbatim}

\includegraphics{rapport_TSP_files/figure-latex/plot res violon-1.pdf}

Rang moyen :

\begin{verbatim}
##   NN best  min  max 
## 3.26 2.52 3.17 1.05
\end{verbatim}

\hypertarget{pour-une-distribution-normale-des-villes}{%
\subsection{Pour une distribution normale des villes
?}\label{pour-une-distribution-normale-des-villes}}

On répète 100 fois les 4 algorithmes sur des données normales générées
par \(\mathcal{N}(0,1) \times \mathcal{N}(0,1)\)

\begin{verbatim}
## No id variables; using all as measure variables
\end{verbatim}

\includegraphics{rapport_TSP_files/figure-latex/plot violon-1.pdf}

Rang moyen :

\begin{verbatim}
##   NN best  min  max 
## 3.44 2.48 3.04 1.04
\end{verbatim}

\textbf{EXERCICE :} Comment évoluent ces résultats si on répète chaque
algorithme pour les \(n\) initialisations possibles?

REPONSE (distribution uniforme):

\begin{verbatim}
## No id variables; using all as measure variables
\end{verbatim}

\includegraphics{rapport_TSP_files/figure-latex/unnamed-chunk-2-1.pdf}

Rang moyen :

\begin{verbatim}
##   NN best  min  max 
## 2.72 2.60 3.66 1.02
\end{verbatim}

\hypertarget{temps-de-calcul}{%
\section{Temps de calcul}\label{temps-de-calcul}}

On étudie ici le temps la complexité des algorithmes en fonction du
nombre \(n\) de villes.

On définit une fonction de type \texttt{one.simu} qui simule une seule
expérience pour un choix de ville.

\begin{Shaded}
\begin{Highlighting}[]
\NormalTok{one.simu\_time\_TSP }\OtherTok{\textless{}{-}} \ControlFlowTok{function}\NormalTok{(i, data, }\AttributeTok{algo =} \StringTok{"NN"}\NormalTok{, }\AttributeTok{type =} \StringTok{"one"}\NormalTok{)}
\NormalTok{\{}
  \ControlFlowTok{if}\NormalTok{(algo }\SpecialCharTok{==} \StringTok{"NN"}\NormalTok{)}
\NormalTok{  \{}
\NormalTok{    start\_time }\OtherTok{\textless{}{-}} \FunctionTok{Sys.time}\NormalTok{()}
    \FunctionTok{NN\_TSP}\NormalTok{(data, }\AttributeTok{type =}\NormalTok{ type)}
\NormalTok{    end\_time  }\OtherTok{\textless{}{-}} \FunctionTok{Sys.time}\NormalTok{()}
\NormalTok{  \}}
  \ControlFlowTok{if}\NormalTok{(algo }\SpecialCharTok{==} \StringTok{"best"}\NormalTok{)}
\NormalTok{  \{}
\NormalTok{    start\_time }\OtherTok{\textless{}{-}} \FunctionTok{Sys.time}\NormalTok{()}
    \FunctionTok{greedy\_TSP\_best}\NormalTok{(data, }\AttributeTok{type =}\NormalTok{ type)}
\NormalTok{    end\_time  }\OtherTok{\textless{}{-}} \FunctionTok{Sys.time}\NormalTok{()}
\NormalTok{  \}}
  \ControlFlowTok{if}\NormalTok{(algo }\SpecialCharTok{==} \StringTok{"min"}\NormalTok{)}
\NormalTok{  \{}
\NormalTok{    start\_time }\OtherTok{\textless{}{-}} \FunctionTok{Sys.time}\NormalTok{()}
    \FunctionTok{greedy\_TSP\_min}\NormalTok{(data, }\AttributeTok{type =}\NormalTok{ type)}
\NormalTok{    end\_time  }\OtherTok{\textless{}{-}} \FunctionTok{Sys.time}\NormalTok{()}
\NormalTok{  \}}
  \ControlFlowTok{if}\NormalTok{(algo }\SpecialCharTok{==} \StringTok{"max"}\NormalTok{)}
\NormalTok{  \{}
\NormalTok{    start\_time }\OtherTok{\textless{}{-}} \FunctionTok{Sys.time}\NormalTok{()}
    \FunctionTok{greedy\_TSP\_max}\NormalTok{(data, }\AttributeTok{type =}\NormalTok{ type)}
\NormalTok{    end\_time  }\OtherTok{\textless{}{-}} \FunctionTok{Sys.time}\NormalTok{()}
\NormalTok{  \}}
  \FunctionTok{return}\NormalTok{(}\FunctionTok{unclass}\NormalTok{(end\_time }\SpecialCharTok{{-}}\NormalTok{ start\_time)[}\DecValTok{1}\NormalTok{])}
\NormalTok{\}}
\end{Highlighting}
\end{Shaded}

On construit un vecteur de taille de ville selon une échelle
logarithmique

\begin{Shaded}
\begin{Highlighting}[]
\NormalTok{my\_n\_vector\_LOG }\OtherTok{\textless{}{-}} \FunctionTok{seq}\NormalTok{(}\AttributeTok{from =} \FunctionTok{log}\NormalTok{(}\DecValTok{10}\NormalTok{), }\AttributeTok{to =} \FunctionTok{log}\NormalTok{(}\DecValTok{100}\NormalTok{), }\AttributeTok{by =} \FunctionTok{log}\NormalTok{(}\DecValTok{10}\NormalTok{)}\SpecialCharTok{/}\DecValTok{40}\NormalTok{)}
\NormalTok{my\_n\_vector }\OtherTok{\textless{}{-}} \FunctionTok{round}\NormalTok{(}\FunctionTok{exp}\NormalTok{(my\_n\_vector\_LOG))}
\NormalTok{my\_n\_vector}
\end{Highlighting}
\end{Shaded}

\begin{verbatim}
##  [1]  10  11  11  12  13  13  14  15  16  17  18  19  20  21  22  24  25  27  28
## [20]  30  32  33  35  38  40  42  45  47  50  53  56  60  63  67  71  75  79  84
## [39]  89  94 100
\end{verbatim}

\begin{Shaded}
\begin{Highlighting}[]
\FunctionTok{diff}\NormalTok{(}\FunctionTok{log}\NormalTok{(my\_n\_vector))}
\end{Highlighting}
\end{Shaded}

\begin{verbatim}
##  [1] 0.09531018 0.00000000 0.08701138 0.08004271 0.00000000 0.07410797
##  [7] 0.06899287 0.06453852 0.06062462 0.05715841 0.05406722 0.05129329
## [13] 0.04879016 0.04652002 0.08701138 0.04082199 0.07696104 0.03636764
## [19] 0.06899287 0.06453852 0.03077166 0.05884050 0.08223810 0.05129329
## [25] 0.04879016 0.06899287 0.04348511 0.06187540 0.05826891 0.05505978
## [31] 0.06899287 0.04879016 0.06155789 0.05798726 0.05480824 0.05195974
## [37] 0.06136895 0.05781957 0.05465841 0.06187540
\end{verbatim}

On construit un data frame qui contiendra les résultats

\begin{Shaded}
\begin{Highlighting}[]
\NormalTok{p }\OtherTok{\textless{}{-}} \DecValTok{50} \DocumentationTok{\#\#\# répétition}
\NormalTok{df }\OtherTok{\textless{}{-}} \FunctionTok{data.frame}\NormalTok{(}\FunctionTok{matrix}\NormalTok{( }\AttributeTok{nrow =} \DecValTok{4} \SpecialCharTok{*} \FunctionTok{length}\NormalTok{(my\_n\_vector), }\AttributeTok{ncol =} \DecValTok{2} \SpecialCharTok{+}\NormalTok{ p))}
\FunctionTok{colnames}\NormalTok{(df) }\OtherTok{\textless{}{-}} \FunctionTok{c}\NormalTok{(}\StringTok{"type"}\NormalTok{, }\StringTok{"n"}\NormalTok{, }\DecValTok{1}\SpecialCharTok{:}\NormalTok{p)}
\FunctionTok{dim}\NormalTok{(df)}
\end{Highlighting}
\end{Shaded}

\begin{verbatim}
## [1] 164  52
\end{verbatim}

On lance la simulation sur plusieurs coeurs.

\begin{Shaded}
\begin{Highlighting}[]
\NormalTok{nbCores }\OtherTok{\textless{}{-}} \DecValTok{8}
\NormalTok{j }\OtherTok{\textless{}{-}} \DecValTok{1}

\ControlFlowTok{for}\NormalTok{(n }\ControlFlowTok{in}\NormalTok{ my\_n\_vector)}
\NormalTok{\{}
\NormalTok{  liste1 }\OtherTok{\textless{}{-}} \FunctionTok{mclapply}\NormalTok{(}\DecValTok{1}\SpecialCharTok{:}\NormalTok{p, }\AttributeTok{FUN =}\NormalTok{ one.simu\_time\_TSP,}
                      \AttributeTok{data =} \FunctionTok{matrix}\NormalTok{(}\FunctionTok{runif}\NormalTok{(}\DecValTok{2}\SpecialCharTok{*}\NormalTok{n), n, }\DecValTok{2}\NormalTok{),}
                     \AttributeTok{algo =} \StringTok{"NN"}\NormalTok{,}
                     \AttributeTok{mc.cores =}\NormalTok{ nbCores)}

\NormalTok{  liste2 }\OtherTok{\textless{}{-}} \FunctionTok{mclapply}\NormalTok{(}\DecValTok{1}\SpecialCharTok{:}\NormalTok{p, }\AttributeTok{FUN =}\NormalTok{ one.simu\_time\_TSP,}
                     \AttributeTok{data =} \FunctionTok{matrix}\NormalTok{(}\FunctionTok{runif}\NormalTok{(}\DecValTok{2}\SpecialCharTok{*}\NormalTok{n), n, }\DecValTok{2}\NormalTok{),}
                     \AttributeTok{algo =} \StringTok{"best"}\NormalTok{,}
                     \AttributeTok{mc.cores =}\NormalTok{ nbCores)}

\NormalTok{  liste3 }\OtherTok{\textless{}{-}} \FunctionTok{mclapply}\NormalTok{(}\DecValTok{1}\SpecialCharTok{:}\NormalTok{p, }\AttributeTok{FUN =}\NormalTok{ one.simu\_time\_TSP,}
                     \AttributeTok{data =} \FunctionTok{matrix}\NormalTok{(}\FunctionTok{runif}\NormalTok{(}\DecValTok{2}\SpecialCharTok{*}\NormalTok{n), n, }\DecValTok{2}\NormalTok{),}
                     \AttributeTok{algo =} \StringTok{"min"}\NormalTok{,}
                     \AttributeTok{mc.cores =}\NormalTok{ nbCores)}
\NormalTok{  liste4 }\OtherTok{\textless{}{-}} \FunctionTok{mclapply}\NormalTok{(}\DecValTok{1}\SpecialCharTok{:}\NormalTok{p, }\AttributeTok{FUN =}\NormalTok{ one.simu\_time\_TSP,}
                     \AttributeTok{data =} \FunctionTok{matrix}\NormalTok{(}\FunctionTok{runif}\NormalTok{(}\DecValTok{2}\SpecialCharTok{*}\NormalTok{n), n, }\DecValTok{2}\NormalTok{),}
                     \AttributeTok{algo =} \StringTok{"max"}\NormalTok{,}
                     \AttributeTok{mc.cores =}\NormalTok{ nbCores)}

\NormalTok{  df[j ,] }\OtherTok{\textless{}{-}} \FunctionTok{c}\NormalTok{(}\StringTok{"NN"}\NormalTok{, n, }\FunctionTok{do.call}\NormalTok{(cbind, liste1))}
\NormalTok{  df[j}\SpecialCharTok{+}\DecValTok{1}\NormalTok{ ,] }\OtherTok{\textless{}{-}} \FunctionTok{c}\NormalTok{(}\StringTok{"best"}\NormalTok{, n, }\FunctionTok{do.call}\NormalTok{(cbind, liste2))}
\NormalTok{  df[j}\SpecialCharTok{+}\DecValTok{2}\NormalTok{ ,] }\OtherTok{\textless{}{-}} \FunctionTok{c}\NormalTok{(}\StringTok{"min"}\NormalTok{, n, }\FunctionTok{do.call}\NormalTok{(cbind, liste3))}
\NormalTok{  df[j}\SpecialCharTok{+}\DecValTok{3}\NormalTok{ ,] }\OtherTok{\textless{}{-}} \FunctionTok{c}\NormalTok{(}\StringTok{"max"}\NormalTok{, n, }\FunctionTok{do.call}\NormalTok{(cbind, liste4))}
\NormalTok{  j }\OtherTok{\textless{}{-}}\NormalTok{ j }\SpecialCharTok{+} \DecValTok{4}
\NormalTok{\}}

\NormalTok{df }\OtherTok{\textless{}{-}} \FunctionTok{melt}\NormalTok{(df, }\AttributeTok{id.vars =} \FunctionTok{c}\NormalTok{(}\StringTok{"type"}\NormalTok{,}\StringTok{"n"}\NormalTok{))}
\end{Highlighting}
\end{Shaded}

tranformations techniques :

\begin{Shaded}
\begin{Highlighting}[]
\NormalTok{data\_summary }\OtherTok{\textless{}{-}} \ControlFlowTok{function}\NormalTok{(data, varname, groupnames)}
\NormalTok{\{}
  \FunctionTok{require}\NormalTok{(plyr)}
\NormalTok{  summary\_func }\OtherTok{\textless{}{-}} \ControlFlowTok{function}\NormalTok{(x, col)}
\NormalTok{  \{}
    \FunctionTok{c}\NormalTok{(}\AttributeTok{mean =} \FunctionTok{mean}\NormalTok{(x[[col]], }\AttributeTok{na.rm=}\ConstantTok{TRUE}\NormalTok{),}
      \AttributeTok{q1 =} \FunctionTok{quantile}\NormalTok{(x[[col]], }\FloatTok{0.025}\NormalTok{), }\AttributeTok{q3 =} \FunctionTok{quantile}\NormalTok{(x[[col]], }\FloatTok{0.975}\NormalTok{))}
\NormalTok{  \}}
\NormalTok{  data\_sum}\OtherTok{\textless{}{-}}\FunctionTok{ddply}\NormalTok{(data, groupnames, }\AttributeTok{.fun=}\NormalTok{summary\_func,}
\NormalTok{                  varname)}
\NormalTok{  data\_sum }\OtherTok{\textless{}{-}} \FunctionTok{rename}\NormalTok{(data\_sum, }\FunctionTok{c}\NormalTok{(}\StringTok{"mean"} \OtherTok{=}\NormalTok{ varname))}
  \FunctionTok{return}\NormalTok{(data\_sum)}
\NormalTok{\}}

\NormalTok{df2 }\OtherTok{\textless{}{-}}\NormalTok{ df}
\NormalTok{df2[,}\DecValTok{2}\NormalTok{] }\OtherTok{\textless{}{-}} \FunctionTok{as.double}\NormalTok{(df[,}\DecValTok{2}\NormalTok{])}
\NormalTok{df2[,}\DecValTok{3}\NormalTok{] }\OtherTok{\textless{}{-}} \FunctionTok{as.double}\NormalTok{(df[,}\DecValTok{3}\NormalTok{])}
\NormalTok{df2[,}\DecValTok{4}\NormalTok{] }\OtherTok{\textless{}{-}} \FunctionTok{as.double}\NormalTok{(df[,}\DecValTok{4}\NormalTok{])}
\FunctionTok{summary}\NormalTok{(df2)}
\end{Highlighting}
\end{Shaded}

\begin{verbatim}
##      type                 n             variable        value          
##  Length:8200        Min.   : 10.00   Min.   : 1.0   Min.   :0.0001569  
##  Class :character   1st Qu.: 18.00   1st Qu.:13.0   1st Qu.:0.0020363  
##  Mode  :character   Median : 32.00   Median :25.5   Median :0.0088866  
##                     Mean   : 39.49   Mean   :25.5   Mean   :0.0481867  
##                     3rd Qu.: 56.00   3rd Qu.:38.0   3rd Qu.:0.0466209  
##                     Max.   :100.00   Max.   :50.0   Max.   :0.8117151
\end{verbatim}

\begin{Shaded}
\begin{Highlighting}[]
\NormalTok{df\_new }\OtherTok{\textless{}{-}} \FunctionTok{data\_summary}\NormalTok{(df2, }\AttributeTok{varname=}\StringTok{"value"}\NormalTok{,}
                           \AttributeTok{groupnames=}\FunctionTok{c}\NormalTok{(}\StringTok{"type"}\NormalTok{,}\StringTok{"n"}\NormalTok{))}
\end{Highlighting}
\end{Shaded}

\begin{verbatim}
## Le chargement a nécessité le package : plyr
\end{verbatim}

\begin{Shaded}
\begin{Highlighting}[]
\NormalTok{theMin }\OtherTok{\textless{}{-}} \FunctionTok{min}\NormalTok{(df\_new[,}\DecValTok{3}\SpecialCharTok{:}\DecValTok{5}\NormalTok{],df\_new[,}\DecValTok{3}\SpecialCharTok{:}\DecValTok{5}\NormalTok{])}
\NormalTok{theMax }\OtherTok{\textless{}{-}} \FunctionTok{max}\NormalTok{(df\_new[,}\DecValTok{3}\SpecialCharTok{:}\DecValTok{5}\NormalTok{],df\_new[,}\DecValTok{3}\SpecialCharTok{:}\DecValTok{5}\NormalTok{])}
\end{Highlighting}
\end{Shaded}

On trace différentes courbes.

\begin{Shaded}
\begin{Highlighting}[]
\FunctionTok{ggplot}\NormalTok{(df\_new, }\FunctionTok{aes}\NormalTok{(}\AttributeTok{x =}\NormalTok{ n, }\AttributeTok{y =}\NormalTok{ value, }\AttributeTok{col=}\NormalTok{type)) }\SpecialCharTok{+}  \FunctionTok{scale\_x\_log10}\NormalTok{()}\SpecialCharTok{+}
  \FunctionTok{scale\_y\_log10}\NormalTok{(}\AttributeTok{limits =} \FunctionTok{c}\NormalTok{(theMin, theMax))  }\SpecialCharTok{+}
  \FunctionTok{labs}\NormalTok{(}\AttributeTok{y =} \StringTok{"time in seconds"}\NormalTok{) }\SpecialCharTok{+}  \FunctionTok{labs}\NormalTok{(}\AttributeTok{x =} \StringTok{"number of cites"}\NormalTok{) }\SpecialCharTok{+}
  \FunctionTok{geom\_point}\NormalTok{(}\AttributeTok{size =} \DecValTok{2}\NormalTok{, }\FunctionTok{aes}\NormalTok{(}\AttributeTok{shape =}\NormalTok{ type)) }\SpecialCharTok{+}
  \FunctionTok{geom\_errorbar}\NormalTok{(}\FunctionTok{aes}\NormalTok{(}\AttributeTok{ymin=}\StringTok{\textasciigrave{}}\AttributeTok{q1.2.5\%}\StringTok{\textasciigrave{}}\NormalTok{, }\AttributeTok{ymax=}\StringTok{\textasciigrave{}}\AttributeTok{q3.97.5\%}\StringTok{\textasciigrave{}}\NormalTok{), }\AttributeTok{width=}\NormalTok{.}\DecValTok{01}\NormalTok{) }\SpecialCharTok{+}
  \FunctionTok{scale\_colour\_manual}\NormalTok{(}\AttributeTok{values =} \FunctionTok{c}\NormalTok{(}\StringTok{"best"} \OtherTok{=} \StringTok{"\#0080FF"}\NormalTok{,}
                                 \StringTok{"max"} \OtherTok{=} \StringTok{" dark blue"}\NormalTok{, }\StringTok{"min"} \OtherTok{=} \StringTok{"blue"}\NormalTok{, }\StringTok{"NN"} \OtherTok{=} \StringTok{"red"}\NormalTok{)) }\SpecialCharTok{+}
  \FunctionTok{theme}\NormalTok{(}\AttributeTok{axis.text.x =} \FunctionTok{element\_text}\NormalTok{(}\AttributeTok{size=}\DecValTok{15}\NormalTok{),}
        \AttributeTok{axis.text.y =} \FunctionTok{element\_text}\NormalTok{(}\AttributeTok{size=}\DecValTok{15}\NormalTok{),}
        \AttributeTok{legend.text=}\FunctionTok{element\_text}\NormalTok{(}\AttributeTok{size=}\DecValTok{15}\NormalTok{),}
        \AttributeTok{axis.title.x=}\FunctionTok{element\_text}\NormalTok{(}\AttributeTok{size=}\DecValTok{15}\NormalTok{),}
        \AttributeTok{axis.title.y=}\FunctionTok{element\_text}\NormalTok{(}\AttributeTok{size=}\DecValTok{15}\NormalTok{),}
        \AttributeTok{legend.position =} \FunctionTok{c}\NormalTok{(}\FloatTok{0.7}\NormalTok{, }\FloatTok{0.1}\NormalTok{),}
        \AttributeTok{legend.title =} \FunctionTok{element\_blank}\NormalTok{())}
\end{Highlighting}
\end{Shaded}

\includegraphics{rapport_TSP_files/figure-latex/plot simu results-1.pdf}

On calcule les valeurs des coefficients directeurs.

Pour NN :

\begin{Shaded}
\begin{Highlighting}[]
\NormalTok{R1 }\OtherTok{\textless{}{-}}\NormalTok{ df\_new[df\_new}\SpecialCharTok{$}\NormalTok{type }\SpecialCharTok{==} \StringTok{"NN"}\NormalTok{,}\FunctionTok{c}\NormalTok{(}\DecValTok{2}\NormalTok{,}\DecValTok{3}\NormalTok{)]}
\NormalTok{l1 }\OtherTok{\textless{}{-}} \FunctionTok{lm}\NormalTok{(}\FunctionTok{log}\NormalTok{(value) }\SpecialCharTok{\textasciitilde{}} \FunctionTok{log}\NormalTok{(n), }\AttributeTok{data =}\NormalTok{ R1, )}
\FunctionTok{summary}\NormalTok{(l1)}
\end{Highlighting}
\end{Shaded}

\begin{verbatim}
## 
## Call:
## lm(formula = log(value) ~ log(n), data = R1)
## 
## Residuals:
##     Min      1Q  Median      3Q     Max 
## -0.4894 -0.3010 -0.1811  0.1088  2.3405 
## 
## Coefficients:
##             Estimate Std. Error t value Pr(>|t|)    
## (Intercept) -11.7878     0.4568  -25.81  < 2e-16 ***
## log(n)        1.6001     0.1281   12.49 7.65e-15 ***
## ---
## Signif. codes:  0 '***' 0.001 '**' 0.01 '*' 0.05 '.' 0.1 ' ' 1
## 
## Residual standard error: 0.5263 on 37 degrees of freedom
## Multiple R-squared:  0.8084, Adjusted R-squared:  0.8032 
## F-statistic: 156.1 on 1 and 37 DF,  p-value: 7.654e-15
\end{verbatim}

\begin{Shaded}
\begin{Highlighting}[]
\NormalTok{l1}\SpecialCharTok{$}\NormalTok{coefficients}
\end{Highlighting}
\end{Shaded}

\begin{verbatim}
## (Intercept)      log(n) 
##  -11.787845    1.600138
\end{verbatim}

Pour best :

\begin{Shaded}
\begin{Highlighting}[]
\NormalTok{R2 }\OtherTok{\textless{}{-}}\NormalTok{ df\_new[df\_new}\SpecialCharTok{$}\NormalTok{type }\SpecialCharTok{==} \StringTok{"best"}\NormalTok{,}\FunctionTok{c}\NormalTok{(}\DecValTok{2}\NormalTok{,}\DecValTok{3}\NormalTok{)]}
\NormalTok{l2 }\OtherTok{\textless{}{-}} \FunctionTok{lm}\NormalTok{(}\FunctionTok{log}\NormalTok{(value) }\SpecialCharTok{\textasciitilde{}} \FunctionTok{log}\NormalTok{(n), }\AttributeTok{data =}\NormalTok{ R2, )}
\FunctionTok{summary}\NormalTok{(l2)}
\end{Highlighting}
\end{Shaded}

\begin{verbatim}
## 
## Call:
## lm(formula = log(value) ~ log(n), data = R2)
## 
## Residuals:
##      Min       1Q   Median       3Q      Max 
## -0.50370 -0.14427 -0.00081  0.08055  0.64943 
## 
## Coefficients:
##              Estimate Std. Error t value Pr(>|t|)    
## (Intercept) -11.59043    0.20120  -57.61   <2e-16 ***
## log(n)        2.37151    0.05642   42.04   <2e-16 ***
## ---
## Signif. codes:  0 '***' 0.001 '**' 0.01 '*' 0.05 '.' 0.1 ' ' 1
## 
## Residual standard error: 0.2318 on 37 degrees of freedom
## Multiple R-squared:  0.9795, Adjusted R-squared:  0.9789 
## F-statistic:  1767 on 1 and 37 DF,  p-value: < 2.2e-16
\end{verbatim}

\begin{Shaded}
\begin{Highlighting}[]
\NormalTok{l2}\SpecialCharTok{$}\NormalTok{coefficients}
\end{Highlighting}
\end{Shaded}

\begin{verbatim}
## (Intercept)      log(n) 
##  -11.590432    2.371513
\end{verbatim}

Pour min :

\begin{Shaded}
\begin{Highlighting}[]
\NormalTok{R3 }\OtherTok{\textless{}{-}}\NormalTok{ df\_new[df\_new}\SpecialCharTok{$}\NormalTok{type }\SpecialCharTok{==} \StringTok{"min"}\NormalTok{,}\FunctionTok{c}\NormalTok{(}\DecValTok{2}\NormalTok{,}\DecValTok{3}\NormalTok{)]}
\NormalTok{l3 }\OtherTok{\textless{}{-}} \FunctionTok{lm}\NormalTok{(}\FunctionTok{log}\NormalTok{(value) }\SpecialCharTok{\textasciitilde{}} \FunctionTok{log}\NormalTok{(n), }\AttributeTok{data =}\NormalTok{ R3, )}
\FunctionTok{summary}\NormalTok{(l3)}
\end{Highlighting}
\end{Shaded}

\begin{verbatim}
## 
## Call:
## lm(formula = log(value) ~ log(n), data = R3)
## 
## Residuals:
##      Min       1Q   Median       3Q      Max 
## -0.48062 -0.15813 -0.00262  0.10151  0.95505 
## 
## Coefficients:
##              Estimate Std. Error t value Pr(>|t|)    
## (Intercept) -11.95913    0.25381  -47.12   <2e-16 ***
## log(n)        2.26161    0.07117   31.78   <2e-16 ***
## ---
## Signif. codes:  0 '***' 0.001 '**' 0.01 '*' 0.05 '.' 0.1 ' ' 1
## 
## Residual standard error: 0.2925 on 37 degrees of freedom
## Multiple R-squared:  0.9647, Adjusted R-squared:  0.9637 
## F-statistic:  1010 on 1 and 37 DF,  p-value: < 2.2e-16
\end{verbatim}

\begin{Shaded}
\begin{Highlighting}[]
\NormalTok{l3}\SpecialCharTok{$}\NormalTok{coefficients}
\end{Highlighting}
\end{Shaded}

\begin{verbatim}
## (Intercept)      log(n) 
##  -11.959131    2.261613
\end{verbatim}

Pour max :

\begin{Shaded}
\begin{Highlighting}[]
\NormalTok{R4 }\OtherTok{\textless{}{-}}\NormalTok{ df\_new[df\_new}\SpecialCharTok{$}\NormalTok{type }\SpecialCharTok{==} \StringTok{"max"}\NormalTok{,}\FunctionTok{c}\NormalTok{(}\DecValTok{2}\NormalTok{,}\DecValTok{3}\NormalTok{)]}
\NormalTok{l4 }\OtherTok{\textless{}{-}} \FunctionTok{lm}\NormalTok{(}\FunctionTok{log}\NormalTok{(value) }\SpecialCharTok{\textasciitilde{}} \FunctionTok{log}\NormalTok{(n), }\AttributeTok{data =}\NormalTok{ R4, )}
\FunctionTok{summary}\NormalTok{(l4)}
\end{Highlighting}
\end{Shaded}

\begin{verbatim}
## 
## Call:
## lm(formula = log(value) ~ log(n), data = R4)
## 
## Residuals:
##      Min       1Q   Median       3Q      Max 
## -0.54206 -0.12442 -0.07636  0.12240  1.21017 
## 
## Coefficients:
##              Estimate Std. Error t value Pr(>|t|)    
## (Intercept) -11.97605    0.30046  -39.86   <2e-16 ***
## log(n)        2.26425    0.08425   26.88   <2e-16 ***
## ---
## Signif. codes:  0 '***' 0.001 '**' 0.01 '*' 0.05 '.' 0.1 ' ' 1
## 
## Residual standard error: 0.3462 on 37 degrees of freedom
## Multiple R-squared:  0.9513, Adjusted R-squared:   0.95 
## F-statistic: 722.3 on 1 and 37 DF,  p-value: < 2.2e-16
\end{verbatim}

\begin{Shaded}
\begin{Highlighting}[]
\NormalTok{l4}\SpecialCharTok{$}\NormalTok{coefficients}
\end{Highlighting}
\end{Shaded}

\begin{verbatim}
## (Intercept)      log(n) 
##  -11.976050    2.264249
\end{verbatim}

\hypertarget{amuxe9lioration-de-tour-par-2-opt-et-3-opt}{%
\section{Amélioration de tour par 2-opt et
3-opt}\label{amuxe9lioration-de-tour-par-2-opt-et-3-opt}}

\textbf{EXERCICE :}

\begin{itemize}
\item
  Ajouter les fonctions \texttt{opt2} et \texttt{opt3} à coder
\item
  Evaluer l'amélioration apportée en terme de distance
\end{itemize}

\hypertarget{algorithme-branch-and-bound}{%
\section{\texorpdfstring{Algorithme \emph{Branch and
Bound}}{Algorithme Branch and Bound}}\label{algorithme-branch-and-bound}}

\begin{itemize}
\item
  Ajouter la fonction \texttt{B\_and\_B}
\item
  Evaluer le coefficient d'approximation des méthodes dans le cas d'une
  répartition uniforme et normale des villes
\item
  Trouver pour un temps maximal donné et un nombre de villes donnés
  quelle est la meilleure méthode à utiliser parmis toutes celles
  proposées !
\end{itemize}

\end{document}
